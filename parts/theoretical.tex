 \chapter{Theoretical Part}
         \section{Classification of Propulsion Systems}

    Propulsion is defined as \textit{"the action or process of propelling"} (\textit{"to drive forward or onward by or as if by means of a force that imparts motion"}). By the Merriam-Webster Dictionary. 

    It can also be defined as "the act of changing the motion of a body with respect to an inertial reference frame."\supercite{sutton2017}

    In engine propulsion, the most common way to achieve such thing is via \textit{chemical combustion}. The energy can also be supplied by \textit{solar radiation}, or a \textit{nuclear reactor}. As such, the varios types of propulsion can be generally divided up into three categories:
    \begin{itemize}
        \item chemical propulsion
        \item nuclear propulsion
        \item solar propulsion
    \end{itemize}

   \begin{table}[H]
     \begin{adjustbox}{minipage=\textwidth+135pt, center}
  \centering
  \begin{threeparttable}
    \begin{tabular}{|| c | c | c | c | c ||}
      \hline \hline
      \multirow{2}{*}{\textbf{Propulsion Device}} &
        \multicolumn{3}{| c |}{\textbf{Energy Source}$^a$} &
        \multirow{2}{*}{\textbf{Propellant or Working Fluid}} \\
        \cline{2-4}
      & Chemical & Nuclear & Solar & \\
      \hline
      Turbojet & D/P &  &  & Fuel + air \\
      \hline
      Turbo-ramjet & TFD &  &  & Fuel + air \\
      \hline
      Ramjet (Hydrocarbon fuel) & D/P & TFD &  & Fuel + air \\
      \hline
      Ramjet ($H_2$ cooled) & TFD &  &  & Hydrogen + air \\
      \hline
      Rocket (chemical) & D/P & TFD &  & Stored propellant \\
      \hline
      Ducted rocket & TFD &  &  & Stored solid fuel + surrounding air \\
      \hline
      Electric rocket & D/P &  & D/P & Stored propellant \\
      \hline
      Nuclear fission rocket &  & TFD &  & Stored $H_2$ \\
      \hline
      Solar heated rocket &  &  & TFD & Stored $H_2$ \\
      \hline
      Photon rocket$^b$ &  & TFND &  & Photon ejection (no stored propellant) \\
      \hline
      Solar sail &  &  & TFD & Photon reflection (no stored propellant) \\
      \hline
    \end{tabular}

    \begin{tablenotes}
      \item[a] \textbf{D/P} developed and/or considered practical; 
      \textbf{TFD} technical feasibility has been demonstrated, but development is incomplete; 
      \textbf{TFND} technical feasibility has not yet been demonstrated.
      \item[b] Essentially a really big light bulb. 
    \end{tablenotes}

    \caption{Energy Sources and Propellants for Various Propulsion Concepts\supercite{sutton2017}}
    \label{tab:1-1}
  \end{threeparttable}
\end{adjustbox}
\end{table} 
  

    Input in rocket propulsion systems is either heat or electricity. Useful output thrust comes from the kinetic energy of the ejected matter and from the propellant pressure on inner chamber walls and at the nozzle exit; thus. rocket propulsion systems primarily convert input energies into the kinetic energy of the exhausted gas. The ejected mass can be in a solid, liquid or gaseous state. Often, combinations of two or more states are ejected. At high enough temperatures, the ejected mass can also be in a state of plasma.\supercite{sutton2017}

    \subsection{Duct Jet Propulsion}
      Duct jet engines, more commonly called "air breathing" engines, are engines which utilize airflow that is then energized inside a duct. They use atmospheric oxygen to burn fuel stored onboard. This class includes the following:\supercite{sutton2017}
      \begin{itemize}
          \item turbojets
          \item turbofans
          \item ramjets
          \item pulse jets
          \item scramjets\supercite{segal2009}
      \end{itemize}
      They are mentioned here mainly as to provide a background and comparison to rocket propulsion engines.
      \begin{table}[H]
        \centering

        \begin{threeparttable}
          \begin{tabular}{|| p{3cm} | p{3cm} | p{3cm} | p{3cm} ||}
              \hline
              Feature & Chemical Rocket Engine or Rocket Motor & Turbojet Engine & Ramjet Engine \\
              \hline\hline
              thrust to weight ratio, typical & 75:1 & 5:1, turbojet and afterburner & 7:1 at Mach 3 at 9,144m (30,000ft) \\
              \hline
              Specific fuel consumption$^{a}$ & 8 - 14 & 0.5 - 1.5 & 2.3 - 3.5 \\
              \hline
              Specific thrust$^{b}$ & 5,000 - 25,000 & 2500 (low Mach$^{c}$ numbers at sea level) & 2700 (Mach 2 at sea level) \\
              \hline
              Specific impulse$^{d}$ & 270 sec & 1600 sec & 1400 sec \\
              \hline
              Thrust change with altitude & Slight increase & Decrease & Decrease \\
              \hline
              Thrust vs. flight speed & Nearly constant & Increases with speed & Increases with speed \\
              \hline
              Thrust vs air temperature & Constant & Decreases with temperaure & Decreases with temperature \\
              \hline
              Flight speed vs. exhaust velocity & Unrelated, flight speed can be greater & Flight speed always less than exhaust velocity & Flight speed always less than exhaust velocity \\
              \hline
              Altitude limitation & None; suited for space travel & 14,000 - 17,000 m & 20,000 m at Mach 3, 30,000 m at Mach 5, 45,000 m at Mach 12 \\
              \hline
          \end{tabular}

          \begin{tablenotes}
            \item[a] Multiply by 0.102 to convert to $kg/(hr-N)$.
            \item[b] Multiply by 47.9 to convert to $N/m^2$
            \item[c] Mach number is the ratio of gas speed to local speed of sound (See Equation~\ref{eq:Mach Number}(Appendix~\ref{app:equations})).
            \item[d] \textit{Specific impulse} is a performance parameter (See Equation~\ref{eq:Specific Impulse}(Appendix~\ref{app:equations})) 
            \end{tablenotes}

          \caption{Comparison of Several Characteristics of a Typical Chemical Propulsion Rocket Propulsion System and Two-Duct Propulsion Systems\supercite{sutton2017}} 
          \label{tab:1-2}
          \end{threeparttable}
        \end{table}


      Out of all of the ducted engines, the \textit{turbojet engine} is the most widely used.

    \subsubsection{Ramjet Engine}
      For supersonic flight in the speeds above Mach 2, the \textit{ramjet engine} (which is a pure ducted engine) is the best suited within the earth's atmosphere. Its compression is purely gas dynamic and thrust is produced by increasing the momentum of the subsonic compressed air as it passes through the ramjet in a very similar manner to the functionality of \textit{turbofan} and \textit{turbojet} engines, just without any compressor or turbine hardware.\supercite{sutton2017} 

      \begin{figure}[H]
        \centering
        \includegraphics[width=0.75\linewidth]{z_Assets/Graphics/Schematics/Ramjet_P280b.jpg}
        \caption{Simplified schematic of a ramjet engine with a supersonic inlet (a converging/diverging flow passage)\supercite{Nasa_ramjet-schematic}}
        \label{fig:ramjet engine}
      \end{figure}

      Ramjets with subsonic combustion and hydrocarbon fuels have an upper speed limit of approx. 5 Mach; Hydrogen fuel with hydrogen cooling raises this maximum to at least 16 Mach.

    \subsubsection{Scramjet Engine}
      The Scramjet engine is a ramjet engine utilizing \textit{super sonic combustion}. Which allows for much freeer and faster air flow than \textit{turbojet} or \textit{ramjet} engines.
      \begin{figure}[H]
        \centering
        \includegraphics[width=0.5\linewidth]{z_Assets/Graphics/Schematics/Turbo_ram_scramjet_comparative_diagram.svg-1.png}
        \caption{The compression, combustion, and expansion regions of: \textbf{(a)} turbojet, \textbf{(b)} ramjet, and \textbf{(c)} scramjet engines.\supercite{Img_scramjet}}
        \label{fig:scramjet engine_comparison}
      \end{figure}
      So far, Scramjet engines have only been used in a few prototype vehicles and military experiments. A Scramjet relies on high vehicle speed to compress the incoming air forcefully before combustion (hence sc (\textit{supersonic combustion}) \textbf{ramjet}), but whereas a ramjet decelerates the air to subsonic velocities before combustion using shock cones, a Scramjet has no shock cone and slows the airflow using shockwaves produced by its ignition source in place of a shock cone.\supercite{segal2009}${}^;$\supercite{Wiki_scramjet}



      \subsection{Rocket Propulsion}
      Rocket propulsion is based fundamentally on \textit{Newton’s Third Law of Motion} — for every action, there is an equal and opposite reaction. In a rocket engine, the “action” is the high-velocity expulsion of exhaust gases, while the “reaction” propels the vehicle forward. Unlike air-breathing engines, rockets carry both fuel and oxidizer, allowing them to function in the vacuum of space.\supercite{sutton2017}

\subsubsection{Chemical Rocket Engines}
Chemical propulsion remains the most widely used method of achieving thrust in spaceflight. It relies on the combustion of chemical propellants that release large amounts of thermal energy, which is converted into kinetic energy of the exhaust gases. Two major types exist: \textit{solid} and \textit{liquid} rocket engines.\supercite{sutton2017}

\paragraph{Solid Rocket Engines}
Solid rocket motors use propellants in a single solid mixture, often consisting of a powdered oxidizer (such as ammonium perchlorate) combined with a binder that also acts as fuel. They are mechanically simple, capable of long storage, and deliver high thrust rapidly after ignition. However, they cannot be throttled or shut down once ignited, limiting flexibility.\supercite{sutton2017}

\begin{figure}[H]
  \begin{center}
    \includegraphics[width=0.5\textwidth]{z_Assets/Graphics/Schematics/Solid-Fuel_Rocket_Diagram.png}
  \end{center}
  \caption{A simplified diagram of a solid-fuel rocket.
    \textbf{(1)} A solid fuel-oxidizer mixture (propellant) is packed into the rocket, with a cylindrical hole in the middle.
    \textbf{(2)} An igniter combusts the surface of the propellant.
    \textbf{(3)} The cylindrical hole in the propellant acts as a combustion chamber.
    \textbf{(4)} The hot exhaust is choked at the throat, which, among other things, dictates the amount of thrust produced.
  \textbf{(5)} Exhaust exits the rocket.
  \supercite{Img_solid-rocket-fuel}
}\label{fig:solid-fuel-rocket-diagram}
\end{figure}

\paragraph{Liquid Rocket Engines}
Liquid-propellant engines store the oxidizer and fuel in separate tanks and feed them into a combustion chamber using pumps or pressurization. The most common combinations are RP-1 (refined kerosene) with liquid oxygen (\ce{LOX}), or liquid hydrogen (\ce{LH2}) with \ce{LOX}. They can be throttled, restarted, and provide high efficiency but require complex plumbing and cryogenic storage.\supercite{sutton2017}

\begin{figure}[H]
  \begin{center}
    \includegraphics[width=0.5\textwidth]{z_Assets/Graphics/Schematics/Liquid-Fuel_Rocket_Diagram.png}
  \end{center}
  \caption{A simplified diagram of a liquid-propellant rocket.
    \textbf{(1)} Liquid rocket fuel.
    \textbf{(2)} Oxidizer.
    \textbf{(3)} Pumps carry the fuel and oxidizer.
    \textbf{(4)} combustion chamber mixes and burns the two liquids.
    \textbf{(5)} Combustion product gasses enter the nozzle through a throat.
  \textbf{(6)} Exhaust exits the rocket.\supercite{Img_liquid-rocket-fuel}}\label{fig:liquid-fuel-rocket-diagram}
\end{figure}

\paragraph{Hybrid Rocket Engines}
A hybrid rocket uses a combination of a liquid oxidizer and a solid fuel. This design offers a compromise between the safety of solid motors and the controllability of liquid systems. Notable examples include SpaceShipOne’s nitrous oxide (\ce{N2O}) and hydroxyl-terminated polybutadiene (\ce{HTPB}) hybrid engine.\supercite{sutton2017}

\begin{figure}[H]
  \begin{center}
    \includegraphics[width=0.75\textwidth]{z_Assets/Graphics/Schematics/SpaceShipOne_schematic.png}
  \end{center}
  \caption{Hybrid rocket motor detail of SpaceShipOne\supercite{Img_spaceshipone}}\label{fig:hybrid-rocket-motor-spaceshipone}
\end{figure}

\subsubsection{Non-Chemical Rocket Engines}
Beyond chemical propulsion, various non-chemical systems have been developed to improve efficiency and endurance for deep-space missions.

\paragraph{Electric Propulsion}
Electric propulsion systems, such as ion and Hall-effect thrusters, use electrical energy (from solar arrays or nuclear sources) to accelerate charged particles. They produce low thrust but extremely high specific impulse, making them ideal for long-duration missions where gradual acceleration is acceptable.\supercite{sutton2017}

\paragraph{Nuclear Thermal Propulsion (NTP)}
In nuclear thermal systems, a nuclear reactor heats a propellant—typically hydrogen—to extremely high temperatures before expansion through a nozzle. This method can theoretically double the specific impulse compared to chemical rockets, offering a promising balance between power and efficiency for interplanetary travel.\supercite{sutton2017}
  \section{Emerging Propulsion Technologies}
    % (Ion engines, nuclear thermal propulsion, solar sails, and antimatter or fusion concepts.)
    As mentioned at the beginning of this work, there still are ideas and possible engines that have yet to depart the zone of science fiction as either our technology isn't advanced enough to produce and test such engines or they have been demonstrated to not be technically effective for their complexity/price or we have no use for them yet. I shall talk about a select few of these in the following parts. Mainly the following:
    \begin{itemize}
      \item Nuclear Propulsion (both Fusion and Fission)
      \item Ion Engines 
      \item Solar Sails
      \item Antimatter Engines
      \item Field Propulsion
    \end{itemize}

    \subsection{Nuclear Propulsion}
    \subsubsection{Fusion}
    \subsubsection{Fission}

    \subsection{Ion Engines}

    \subsection{Solar Sails}

    \subsection{Antimatter Engines}

    \subsection{Field Propulsion}
  Field propulsion, as defined by Yoshinari Minami is the act of propulsion not by usual means of momentum thrust, but instead by pressure thrust derived from an interaction of a spaceship with the physical structure of space-time. (Assuming that space as a vacuum possesses a substantial physical structure.)\supercite{minami2018}\\

    As the theory stands, field propulsion remains speculative and has not yet been experimentally demonstrated. It arises from the idea that spacetime itself may be manipulated to create a net force on a spacecraft without the expulsion of reaction mass. Concepts such as the \textit{Alcubierre warp drive}, derived from solutions to Einstein’s field equations, suggest that spacetime could theoretically be contracted in front of and expanded behind a spacecraft, allowing apparent faster-than-light travel.\supercite{minami2018}

    However, the energy requirements are currently astronomical, exceeding the mass–energy of entire planets, and the feasibility of generating negative energy densities remains purely theoretical. Despite these limitations, research into quantum vacuum interactions and general relativistic field manipulation continues at a conceptual level, keeping the idea alive in advanced propulsion discussions.\supercite{minami2018}


  \section{Theoretical Framework for Interstellar Propulsion}
    Interstellar travel demands propulsion technologies far beyond current chemical or even nuclear capabilities. Several theoretical and conceptual projects have proposed methods to achieve fractions of the speed of light.

\subsection{Project Daedalus}
Conceived in the 1970s by the British Interplanetary Society, \textit{Project Daedalus} proposed a two-stage fusion-powered spacecraft capable of reaching 12\% the speed of light. It would use pellets of deuterium and helium-3 ignited by electron beams to produce thrust. Though technologically beyond current reach, Daedalus provided a credible engineering framework for interstellar flight.\supercite{project_daedalus_propulsion}

\subsection{Breakthrough Starshot}
A modern descendant of the Daedalus concept, \textit{Breakthrough Starshot} envisions launching gram-scale probes accelerated by ground-based laser arrays to 20\% the speed of light. The probes would use lightweight sails reflecting focused laser beams, allowing rapid travel to nearby stars like Proxima Centauri within a human lifetime.\supercite{Wiki_breakthrough_starshot}

\subsection{The Alcubierre Drive}
Proposed by Miguel Alcubierre in 1994, the \textit{warp drive} concept relies on the manipulation of spacetime geometry, compressing space ahead of the craft and expanding it behind. This would allow effective faster-than-light travel without violating local relativistic constraints. Nonetheless, it requires exotic matter with negative energy density, which has not been observed in usable quantities.\supercite{alcubierre1994}
\\

These speculative frameworks demonstrate the profound link between physics and engineering in the pursuit of interstellar travel, emphasizing that humanity’s ultimate propulsion systems may rely as much on breakthroughs in fundamental science as on technological advancement.


