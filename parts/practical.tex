\chapter{Practical Part}
   \section{Methodics}
In this practical part I will use the software  \textit{Numerical Propulsion System Simulation} (NPSS) to design a liquid‑propellant rocket engine.  
The aim is to follow a clear, ordered method so that the engine model is built, simulated, and analysed step by step.

\begin{enumerate}
  \item \textbf{Define the design requirements}
    \begin{itemize}
      \item Choose the engine’s purpose (for example: first stage of a launch vehicle, sea‑level take‑off).  
      \item Select the propellants (for example RP‑1 + LOX, or \ce{LH2} + LOX).  
      \item Set the main targets: thrust (in Newtons or kN), burn time (seconds), maximum allowable engine mass, and operating environment (sea level vs vacuum).
    \end{itemize}

  \item \textbf{Select the engine cycle \& make initial assumptions} 
    \begin{itemize}
      \item Decide on the engine cycle type (gas‑generator, staged‑combustion, etc.).  
      \item Make initial assumptions: chamber pressure, mixture ratio (oxidiser : fuel), expansion ratio of the nozzle.  
      \item Note constraints: engine size, material limits, cooling, manufacturability.
    \end{itemize}

  \item \textbf{Build the NPSS model}
    \begin{itemize}
      \item Set up NPSS with the main components: feed system, pumps/turbines, combustion chamber, nozzle.  
      \item Input the assumed parameters: mass flows, pressures, temperatures, geometry (for example throat area, exit area).  
      \item Run the simulation for the “design‑point” (the normal operating condition).
    \end{itemize}

  \item \textbf{Analyse the design‑point results}
    \begin{itemize}
      \item Extract performance data: thrust, specific impulse ($I_{\!sp}$), propellant mass flow, chamber and nozzle pressures and temperatures.  
      \item Compare the results with your design targets from Step 1. If the targets are not met, return to Step 2 (adjust assumptions) or Step 3 (modify model) and iterate.
    \end{itemize}

  \item \textbf{Perform parametric / sensitivity studies}
    \begin{itemize}
      \item Vary one parameter at a time (for example mixture ratio, chamber pressure, expansion ratio) and see how performance changes.  
      \item Record which parameters have the largest effect (sensitivity) and find trade‑offs (for example higher chamber pressure improves $I_{\!sp}$ but may increase mass or cost).  
      \item Use these findings to refine your design toward a better balance.
    \end{itemize}

  \item \textbf{Check off‑design / operating range conditions}
    \begin{itemize}
      \item Use NPSS to simulate the engine under different conditions: for example different ambient pressures (sea level vs high altitude), partial throttle or start‑up conditions.  
      \item Determine how performance changes under these conditions: does thrust drop? does $I_{\!sp}$ fall? Are there stability problems?
    \end{itemize}

  \item \textbf{Document and reflect on the results}
    \begin{itemize}
      \item Prepare tables and graphs presenting your results (design point outputs, parametric study results).  
      \item Discuss your assumptions (for example: ignoring cooling losses, assuming ideal pumps) and their impact.  
      \item Reflect on what I would do differently if you had more time or resources, and what the limits of my design are.
    \end{itemize}
\end{enumerate}
