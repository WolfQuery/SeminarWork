\chapter{Practical Part}
\section{Methodics}
For this \textit{experiment} to actually be fruticious, we must first define goals and limitations we are aiming to achieve. Those will be the following:
\subsection{Goals}
\begin{enumerate}
  \item To design a liquid-propellant rocket engine block capable of delivering performance metrics that correspond to a $\Delta v$ on the order of $~1 \%$ of the speed of light ($\approx 2.997\times 10^6  m/s)$.  
  \item To specify and simulate the engine cycle, propellant combination, and nozzle geometry using Numerical Propulsion System Simulation (NPSS) so that key performance parameters (thrust, specific impulse, effective exhaust velocity, mass-flow) are extracted and analysed.  
  \item To perform a parametric or sensitivity study varying key independent variables (e.g., chamber pressure, mixture ratio, nozzle expansion ratio) and determine which parameters most significantly affect engine performance.  
  \item To evaluate off-design conditions (e.g., different ambient pressures, throttle levels) to assess how robust the engine design is across a realistic range of operating environments.
\end{enumerate}

\subsection{Limitations \& Assumptions}
This simulation-study is subject to the following limitations and assumptions:
\begin{itemize}
  \item Steady-state operation is assumed for the engine block; transient behaviours such as start-up, shutdown, or rapid throttle changes are not modelled.  
  \item The structural mass of the engine block is assumed constant and does not scale with thrust or size; in reality larger thrust engines would incur higher structural mass which is not captured.  
  \item Detailed mechanical losses in the feed-system (e.g., pump/turbine leakages, bearing losses) are approximated by fixed efficiency values rather than fully modelled computationally.  
  \item The study focuses solely on the engine block (feed‐system, combustion chamber, nozzle) and \textbf{does not} model full vehicle design, staging or complete mission $\Delta v$ budgets.  
  \item Propellant storage, tankage, logistics and launch infrastructure (e.g., space‐elevator, in-space construction) are outside the scope of this work and only referenced conceptually.  
  \item The mission target (~1\% c) is conceptual and serves to guide the performance targets; it is \textbf{not} assumed this engine would alone enable a vehicle to reach interstellar travel without further system stages or infrastructure.
\end{itemize}

\section{Hypothesis}
\section{Hypothesis}

It is hypothesized that by scaling the thrust and performance parameters of the baseline SpaceX Raptor 2 engine (vacuum thrust $\approx2.53 MN$) by a factor of 100, the resulting conceptual liquid-propellant engine block will enable a $\Delta v$ budget corresponding to approximately 1\% of the speed of light (i.e., $\approx 2.997\times10^{6}\;\mathrm{m/s}$) under optimised deep-space vacuum conditions.

The simulation in Numerical Propulsion System Simulation (NPSS) will demonstrate that the chosen engine cycle, propellant combination, nozzle geometry and mass-flow performance yield a specific impulse, effective exhaust velocity and thrust level sufficient to achieve the target $\Delta v$, and that a parametric study of chamber pressure, mixture ratio and nozzle expansion ratio will identify the key parameters most significantly affecting the engine’s performance.

This will allow us to calculate the approximate size and fuel requirement for this theoretical engine utilizing Tsiolkovsky's Rocket Equation.

For these assummptions:
\begin{itemize}
  \item Wet to dry mass ratio ($\frac{m_0}{m_f}$) of 20 (i.e. the rocket carries 20 times its mass in fuel initially).
  \item $\Delta v = 2.997\times10^6 m/s$
\end{itemize}
This allows us to calculate the effective exhaust velocity we must reach:
$$
v_e = \frac{2.997\times10^6}{\ln(20)}\approx\frac{3\times10^6}{2.99573227355}=1001424.60209\approx 1,001,425\ m/s = 3,605,130\ km/h
$$
In this setup the gas exhaust velocity would have to be $\approx 30 \text{ mach}$ to reach $1\%c$ once all of the fuel is exhausted (assuming that the rocket carries 20 times its dry weight in fuel.
\section{Design}

