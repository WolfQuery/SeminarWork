\chapter{Conclusion}

\section{Overview}
The aim of this work was to explore how rocket engines have developed over time and how their evolution shapes humanity’s prospects for interstellar travel. The thesis combined three parts: a historical overview, a theoretical explanation of propulsion systems and rocket physics, and a conceptual analysis of what would be required for a rocket engine to reach roughly 1\% of the speed of light.

\section{Key Findings}

\subsection{Historical Perspective}
From early gunpowder rockets to the V-2, Saturn V, and modern engines such as Merlin and Raptor, the history of rocket propulsion shows steady progress toward higher efficiency, improved reliability and, more recently, reusability. Each technological step broadened the range of missions achievable by human-made spacecraft.

\subsection{Theoretical Insights}
The theoretical section showed that propulsion systems differ greatly in capability and purpose. Chemical engines remain unmatched for launch, yet their specific impulse is ultimately limited by chemistry. Concepts such as nuclear propulsion, electric engines, solar sails or even field propulsion offer far higher potential efficiencies, but most remain technically immature or purely speculative. The Tsiolkovsky equation clearly illustrates the core challenge: achieving high $\Delta v$ requires either extremely high exhaust velocities or exponential growth in propellant mass.

\subsection{Conceptual Analysis}
The conceptual analysis applied these principles to the target of achieving a $\Delta v$ of approximately $1\%c$. Even under simplified assumptions and a favourable mass ratio, the required exhaust velocity was found to be on the order of one million metres per second. Scaling a modern chemical engine, such as the Raptor, by any realistic factor does not approach these velocities. This reinforces the conclusion that chemical propulsion cannot serve as a basis for interstellar travel.

\section{Evaluation of Objectives}
Although the initial plan envisioned a simulated practical design, the conceptual approach proved sufficient to meet the core objectives: reviewing historical developments, explaining propulsion theory, and assessing the feasibility of high‐velocity travel. The simplified analysis still demonstrates the physical limits of current engine technology in a clear and meaningful way.

\section{Limitations and Future Work}
The study relied on idealised assumptions, steady-state operation and simplified mass models. No full vehicle architecture or detailed engine cycles were evaluated. Future work could focus on realistic modelling of nuclear or beamed propulsion concepts, or on examining the engineering challenges behind high-energy propulsion systems.

\section{Final Remarks}
Rocket propulsion has taken humanity from simple fireworks to reusable orbital launch vehicles in a historically short time. Yet interstellar travel remains far beyond the reach of current technology. The path forward will depend on breakthroughs in physics and energy generation rather than incremental improvements to chemical engines. Even so, the steady progress of propulsion engineering gives reason to believe that the dream of reaching the stars may eventually move from theory to reality.

