% soctepmlate unofficial - SOČ = Středoškolská odborná činnost - Czech competition
% Author: Vojtěch Boček
% Edit by: Jaroslav Páral
% Version: 2018-02-12
% Source code: https://github.com/RoboticsBrno/soctemplate/
% Base on: http://www.jcmm.cz/cz/sablona-soc.html
% License: CC BY 4.0

\documentclass{template/socthesis}

\usepackage{subcaption}
\usepackage{amsmath}
\usepackage{enumitem}

\addbibresource{text.bib}

\titlecz{Experimentální studium valivých pohybů a~jejich dopadů na~renesančního člověka pohledem Immanuela Kanta}
\titleen{Experimental Study of Rolling Motions and Their Impact on the Renaissance Man from the Perspective of Immaniel Kant}
\author{Jan Novák}
\field{14}
\school{Gymnázium Brno, třída kpt.~Jaroše}
\mentor{doc. PhDr. Jana Nováková, Ph.D.}
\mentorstatement{doc. PhDr. Jany Novákové, Ph.D.}

% Změňte, pokud se liší
%\region{Jihomoravský}
%\placefooter{Brno 2017}

\begin{document}
	
	\maketitle
	
	\makecopyrightstatement{V~Brně}
	
	\makethanks{Děkuji své školitelce doc. PhDr. Janě Novotné, Ph.D. za obětavou pomoc, podnětné připomínky a nekonečnou trpělivost, kterou mi během práce poskytovala.}
	
	\pagestyle{empty}
	
	\section*{Anotace}
	Cílem této práce je vytvořit univerzální dálkový ovládací pult, který se od běžně dostupných ovladačů liší tím, že umožňuje uživateli rozmístit si libovolně a v~téměř neomezeném množství ovládací prvky ze standardní nabídky modulů.
	
	\subsection*{Klíčová slova}
	univerzální ovladač; řídicí pult; dálkové ovládání; komunikace; modulární konstrukce
	
	\vspace{20mm}
	
	\section*{Annotation}
	The goal of this work is to create a multi-purpose remote control console.
	Unique feature of this console is the possibility for user to place any and almost unlimited amount of operating ele-ments (from range of standard, premade modules) wherever he needs to and in any layout he wants.
	
	\subsection*{Keywords}
	universal controller; control board; remote control; communication; modular construction
	
	\newpage
	\pagestyle{plain}
	
    \tableofcontents % vysází obsah
	
	%%% Začátek práce
	\setcounter{figure}{0}
	\setcounter{table}{0}
	\newpage
	
	%%% Úvod
	\input{text-01-uvod.tex}
	
	%%% Jak psát
	\chapter{Jak psát}
Abychom mohli napsat odborný text jasně a srozumitelně, musíme splnit několik základních předpokladů\cite{vut-zkousky}:
\begin{itemize}
	\item musíme mít co říci,
	\item musíme vědět, komu to chceme říci,
	\item musíme si dokonale promyslet obsah,
	\item musíme psát strukturovaně.
\end{itemize}

\section{Musíme mít co říci}
Nejdůležitějším předpokladem dobrého odborného textu je myšlenka.
Je-li myšlenka dost závažná, tak přetrvá, i když je neobratně a zmateně podaná.
Chceme-li však myšlenku podat co nejvýstižněji a ušetřit tak čtenáři čas, musíme dodržet určité zásady, o~kterých pojednáme dále.

\section{Musíme vědět, komu to chceme říci}
Dalším důležitým předpokladem dobrého psaní je psát pro někoho.
Píšeme-li si poznámky sami pro sebe, píšeme je jinak než výzkumnou zprávu, článek, diplomovou práci, knihu nebo dopis.
Podle předpokládaného čtenáře se rozhodneme pro způsob psaní, rozsah informace a míru detailů.

\section{Musíme si dokonale promyslet obsah}
Jakmile víme, co chceme říci a komu, musíme si rozvrhnout látku.
Ideální je takové rozvržení, které tvoří logicky přesný a psychologicky stravitelný celek, ve kterém je pro všechno místo a jehož jednotlivé části do sebe přesně zapadají.
Jsou jasné všechny souvislosti a je zřejmé, co kam patří.

Abychom tohoto cíle dosáhli, musíme pečlivě organizovat látku.
Rozhodneme, co budou hlavní kapitoly, co podkapitoly a jaké jsou mezi nimi vztahy.
Diagramem takové organizace je graf, který je velmi podobný stromu, ale ne řetězci.
Při organizaci látky je stejně důležitá otázka, co do osnovy zahrnout, jako otázka, co z~ní vypustit.
Příliš mnoho podrobností může čtenáře právě tak odradit jako žádné detaily.

\section{Musíme začít psát strukturovaně}
Máme-li tedy myšlenku, představu o~budoucím čtenáři, cíl a osnovu textu, můžeme začít psát.
Při psaní prvního konceptu se snažíme zaznamenat všechny své myšlenky a názory vztahující se k~jednotlivým kapitolám a podkapitolám.
Každou myšlenku musíme vysvětlit, popsat a prokázat.
Hlavní myšlenku má vždy vyjadřovat hlavní věta a nikoliv věta vedlejší.

	
	%%% Několik formálních pravidel
	\chapter{Několik formálních pravidel}
Naším cílem je vytvořit jasný a srozumitelný text.
Vyjadřujeme se proto přesně, píšeme dobrou češtinou (nebo zpravidla angličtinou) a dobrým slohem podle obecně přijatých zvyklostí.
Text má upravit čtenáři cestu k~rychlému pochopení problému, předvídat jeho obtíže a předcházet jim.
Dobrý sloh předpokládá \B{bezvadnou gramatiku, správnou interpunkci a vhodnou volbu slov.} Snažíme se, aby náš text nepůsobil příliš jednotvárně používáním malého výběru slov a tím, že některá zvlášť oblíbená slova používáme příliš často.
Pokud používáme cizích slov, je samozřejmým předpokladem, že známe jejich přesný význam.
Ale i českých slov musíme používat ve správném smyslu.
Např.
platí jistá pravidla při používání slova zřejmě.
Je zřejmé opravdu zřejmé? A~přesvědčili jsme se, zda to, co je zřejmé opravdu platí? Pozor bychom si měli dát i na příliš časté používání zvratného se.
Například obratu dokázalo se, že… zásadně nepoužíváme.
Není špatné používat autorského my, tím předpokládáme, že něco řešíme, nebo například zobecňujeme spolu se čtenářem.
V~kvalifikačních pracích použijeme autorského já (například když vymezujeme podíl vlastní práce vůči převzatému textu), ale v~běžném textu se nadměrné používání první osoby jednotného čísla nedoporučuje.

Za pečlivý výběr stojí i \B{symbolika}, kterou používáme ke značení.
Máme tím na mysli volbu zkratek a symbolů používaných například pro vyjádření typů součástek, pro označení hlavních činností programu, pro pojmenování ovládacích kláves na klávesnici, pro pojmenování proměnných v~matematických formulích a podobně.
Výstižné a důsledné značení může čtenáři při četbě textu velmi pomoci.
Je vhodné uvést seznam značení na začátku textu.
Nejen ve značení, ale i v~odkazech a v~celkové tiskové úpravě je důležitá důslednost.

S~tím souvisí i pojem z~typografie nazývaný \B{vyznačování}.
Zde máme na mysli způsob sazby textu pro jeho zvýraznění.
Pro zvolené značení by měl být zvolen i způsob vyznačování v~textu.
Tak například klávesy mohou být umístěny do obdélníčku, identifikátory ze zdrojového textu mohou být vypisovány písmem typu psací stroj.

Uvádíme-li některá fakta, neskrýváme jejich původ a náš vztah k~nim.
Když něco tvrdíme, vždycky výslovně uvedeme, co z~toho bylo dokázáno, co teprve bude dokázáno v~našem textu a co přebíráme z~literatury s~uvedením odkazu na příslušný zdroj.
V~tomto směru nenecháváme čtenáře \B{nikdy na pochybách}, zda jde o~myšlenku naši nebo převzatou z~literatury.

Nikdy neplýtváme čtenářovým časem výkladem triviálních a nepodstatných informací.
Neuvádíme rovněž několikrát totéž jen jinými slovy.
Při pozdějších úpravách textu se nám může některá dříve napsaná pasáž jevit jako zbytečně podrobná, nebo dokonce zcela zbytečná.
Vypuštění takové pasáže nebo alespoň její zestručnění přispěje k~lepší čitelnosti práce! Tento krok ale vyžaduje odvahu zahodit čas, který jsme jejímu vytvoření věnovali.
	
	
	%%% Nikdy to nebude naprosto dokonalé
	\input{text-04-nebude-to-dokonale.tex}
	
	%%% Typografické a jazykové zásady
	\input{text-05-typograficke-a-jazykove-zasady.tex}
	
	%%% Slovo Romana
	\input{text-06-slovo-Romana.tex}
	
	%%% Slovo Jarka
	\input{text-07-slovo-Jarka.tex}
	
	%%% Slovo Lucie
	\input{text-08-slovo-Lucie.tex}
	
	%%% Závěr
	\input{text-09-zaver.tex}
	
	\newpage
	\printbibliography[title=Literatura]
	\addcontentsline{toc}{section}{Literatura}
	
	\listoffigures
	\addcontentsline{toc}{section}{Seznam obrázků}
	
	\listoftables
	\addcontentsline{toc}{section}{Seznam tabulek}
	
	\listoflistedequation
	\addcontentsline{toc}{section}{Seznam rovnic}
	
\end{document}
